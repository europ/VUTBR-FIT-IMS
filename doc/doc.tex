\documentclass[11pt,a4paper]{article}

\usepackage[left=2cm,text={17cm,24cm},top=3cm]{geometry}
\usepackage[bookmarksopen,colorlinks,plainpages=false,urlcolor=blue,unicode,linkcolor=blue]{hyperref}
\usepackage[slovak]{babel}
\usepackage[utf8]{inputenc}
\usepackage[T1]{fontenc}
\usepackage{indentfirst}
\usepackage{graphicx}
\usepackage{eurosym}
\usepackage{url}

\graphicspath{{img/}}

\begin{document}

% #################################################################################################
% TITLEPAGE

\begin{titlepage}
    \begin{center}
        \Huge
        \textsc{
            Fakulta informačních technologií\\
            Vysoké učení technické v~Brně
        }
        \vspace{100px}
        \begin{figure}[!h]
            \centering
            \includegraphics[scale=0.3]{vutbr-fit-logo.eps}
        \end{figure}
        \\[25mm]
        \huge{
            \textbf{
                Odvoz komunálneho odpadu v meste\\
                Rimavská Sobota
            }
        }
        \vfill
    \end{center}
        \Large{
            \hfill\\
            Peter Šuhaj (xsuhaj02)\\
            Adrián Tóth (xtotha01) \hfill \today
        }

\end{titlepage}

% #################################################################################################
% CONTENT

\setlength{\parskip}{0pt}
\hypersetup{hidelinks}\tableofcontents
\setlength{\parskip}{0pt}

\newpage %#########################################################################################

\section{Úvod}

    \indent Táto práca sa zoberá problematikou odvozu komunálneho odpadu v meste Rimavská Sobota v Slovenskej republike. Pre daný problém bol navrhnutý a implementovaný model - \cite{IMS}(str. č. 7), ktorý sa využíva na simulačné účely - \cite{IMS}(str. č. 8). Simulačné výsledky slúžia na získavanie poznatkov o odvoze komunálneho odpadu, ktoré budú skúmané a popísane v ďalších kapitolách.\\[0.4em]
    \indent Cieľom práce je poukázať na fakty ktoré môžu odvoz komunálneho odpadu ovplivniť z ekonomického hľadiska. Zmysel tejto práce je nájsť čo najekonomickejšie riešenie na náklady odvozu odpadu v meste Rimavská Sobota.

    \subsection{Zadanie}

        \indent Zadanie tejto práce spadalo do časti \textit{služby, infrastruktura a energetika}\cite{IMS-TEMA} pod číslom \textit{5}, ktoré nám bolo náhodne pridelené. Vybrali sme si problematiku \textit{odpadové hospodárstvo} z oblasti \textit{infraštruktúry}.

    \subsection{Autori a zdroje}

        \indent Autormi tohto projektu sú Peter Šuhaj (xsuhaj02) a Adrián Tóth (xtotha01) - študenti 3. ročníka bakalárskeho štúdia na \textit{Fakultě informačních technologií - Vysokého učení technického v Brně}\cite{VUT-FIT}.\\[0.4em]
        \indent Zdroje informácii potrebné k vypracovaniu projektu boli získané od pána Attilu Šimka ktorý je vedúci odboru zodpovedného pre odvoz komunálneho odpadu pre mesto Rimavská Sobota nachádzajúce sa na území Slovenskej Republiky. Pán A. Šimko nám poskytol konzultáciu a odborné fakty na základe ktorých sme mohli navrhnúť náš model - \cite{IMS}(str. č. 7).

    \subsection{Validita modelu}

        \indent Validitu modelu - \cite{IMS}(str. č. 37), sa nám podarilo overiť na základe konfrontácie odborných faktov a faktov získaných zo simulačných výsledkov pomocou mnohých experimentov. Pomocou tejto konfrontácie sme mohli klasifikovať náš model za validný t.j. model adekvátny modelovanému systému - \cite{IMS}(str. č. 7).

\section{Rozbor témy}

    \indent Potrebné informácie a fakty potrebné k implementácii boli získané na osobnom stretnutí s pánom Attilom Šimkom v organizácii \textit{Technické služby mesta Rimavská Sobota}\cite{TSMRS}. Na osobnom stretnutí nám boli poskytnuté štatistické informácie o komunálnom odpade, informácie o vozidlách potrebných na odvoz komunálneho odpadu a o skládke na komunálny odpad.\\[0.4em]
    \indent Štatistické informácie o komunálnom odpade poskytnuté pánom A. Šimkom boli nasledovné: typ a počet smetných košov na uliciach, priemerná váha celkového odpadu v tonách vyzbieraného za jeden deň. Priemerný čas potrebný ku spracovaniu jedného smetného koša bol získaný osobním meraním v teréne.\\[0.4em]
    \indent Stroje odvážajúce komunálny odpad sú značky Mercedes-Benz typu Econic s nosnosťou 10,4 ton. Priemerná spotreba jedného vozidla je 70 litrov nafty na 100 kilometrov. Rýchlosť vozidla počas zberu smetí, t.j. pohyb vozidla medzi smetnými kôšmi na ulici a pri presune medzi jednotlivými ulicami, je v priemere 20km/h. Všetky informácie o vozidlách, ktoré odvážajú komunálny odpad, nám boli poskynuté na osobnom stretnutí od pána A. Šimka, keďže sa jedná o vozidlá ktoré boli zakúpené organizáciou v Nemecku ako ojazdené vozidlá.\\[0.4em]
    \indent Zber komunálneho odpadu v meste Rimavská Sobota sa delí na zóny\cite{ZONA}, kde sa za jeden deň vyzbierajú 2 zóny pričom do zón sa vyšlú dve autá. Jedno auto má na starosti vyzbierať komunálny odpad jednej zóny za jeden deň ktorá mu je pridelená.\\[0.4em]
    \indent Cena paliva k behu vozidla je 1.20\euro{} za jeden liter nafty\cite{NAFTA}. Vzdialenosť skládky od mesta je 26 až 28 kilometrov v závislosti od polohy auta v meste\cite{VZDIALENOST}.

    \subsection{Použité postupy}

        Jazyk C++ sme zvolili z dôvodu že existuje vytvorená knižnica na simulovanie pre tento jazyk, ďalej je prenositeľný, rýchly, imperatívny a objektovo orientovaný. Knižnica SIMLIB bola zvolená pretože ponúka prostriedky na implementáciu simulačného modelu - \cite{IMS}(str. č. 44), a tým zjednodušuje jeho implementáciu.

    \subsection{Použité technológie}

        \noindent K implementácii boli využité nasledovné technológie:
        \begin{itemize}
            \item C++ {\color{blue}{\href{http://www.cplusplus.com/}{www.cplusplus.com}}}
            \item g++ {\color{blue}{\href{https://www.cprogramming.com/g++.html}{www.cprogramming.com/g++.html}}}
            \item SIMLIB {\color{blue}{\href{http://www.fit.vutbr.cz/\~peringer/SIMLIB/}{www.fit.vutbr.cz/{$\sim$}peringer/SIMLIB}}}
            \item Ubuntu 16.04.3 LTS {\color{blue}{\href{http://releases.ubuntu.com/16.04/}{releases.ubuntu.com/16.04}}}
        \end{itemize}

        \indent Rozhodli sme sa pre výber hore uvedených technológii pretože sú voľne dostupné a súčastne je možné pomocou nich implementovať simulačný model - \cite{IMS}(str. č. 44). Jazyk C++ a prekladač g++ sme sa rozhodli použiť kvôli knižnici SIMLIB, ktorá je vytvorená v jazyku C++.

\section{Koncepcia modelu}

    \indent Táto práca sa zaobéra s odvozom komunálneho odpadu, t.j. zozbieranie komunálneho odpadu a jeho vývoz na skládku.\\[0.4em]
    \indent V rámci tohto modelu - \cite{IMS}(str. č. 7), vychádzame zo zdrojov spomenutých v kapitole 2 \textit{Rozbor témy} ktoré sa vzťahujú na náklady zberu a vývozu komunálneho odpadu. Bolo zanedbané zrýchlenie vozidla z dôvodu používania priemernej rýchlosti vozidla. Náklady zberu a vývozu odpadu nesúvisia s nákladmi na správu vozidiel, správa a prevádzka skládky, takže tieto náklady sú tiež zanedbané. Jediná súvislosť so skládkou je jej vzdialenosť, ktorá má výrazný vplyv na náklady potrebné na odvoz odpadu.\\[0.4em]
    \indent Vozidlo má vopred stanovené ulice ktoré má vyzbierať za daný deň. Doba presunu vozidla počas zberu odpadu medzi ulicami je zanedbaná, keďže ulice sú medzi sebo prepojené.\\[0.4em]
    \indent Pre popis modelu - \cite{IMS}(str. č. 7), sme použili deklaratívny model typu petriho siete - \cite{IMS}(str. č. 49).

    \subsection{Forma konceptuálneho modelu}

        Asdf.


\section{Architektúra simulačného modelu}

    Asdf.

    \subsection{Návrh}

        Asdf.

\section{Simulačné experimenty a ich priebeh}

    Asdf.

    \subsection{Postup simulačných experimentovaní}

        Asdf.

    \subsection{Jednotlivé simulačné experimenty}

        Asdf.

    \subsection{Záver simulačných experimentovaní}

        Asdf.

\section{Záver}

    Asdf.



\newpage %#########################################################################################

\makeatletter
\makeatother
\bibliographystyle{czechiso}
\begin{flushleft}
    \bibliography{quotation}
\end{flushleft}

\end{document} %###################################################################################


